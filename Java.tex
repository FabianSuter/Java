\documentclass[fontsize=8pt, a4paper, fleqn, landscape, DIV=calc]{scrartcl}
\usepackage[utf8]{inputenc}
\usepackage[ngerman]{babel}
\usepackage{tikz}   % M.K.: added knma

\usepackage{hyperref}           % Links in the pdf
\hypersetup{pdfborder = {0 0 0}}% No borders around Hyperlinks

%Layout
\usepackage{multicol,   % Mehrere Spalten
            geometry,   % Page layout
            tcolorbox,  % Color boxes for titles and sections
            fancyhdr,   % Header / Footer
            lastpage,   % Link to last page
            dirtytalk,  % Make the use "" easier
            qrcode,     % QR-Code im Titel
            rotating,   % Rotate tables
            tabularx}   % Fancier tables, mainly auto-text-wrap

\geometry{margin=1cm}
\parindent 0pt
\pagestyle{fancy}
\newlength{\breite}
\setlength{\breite}{0.5pt}
\setlength{\columnseprule}{\breite}

\usepackage[table,xcdraw]{xcolor}% color

%Math stuff
\usepackage{mathtools}
%\allowdisplaybreaks %allow display breaks in align

%automate Columnbreak
\newcommand{\nextcol}{%
\vfill\null%
\columnbreak%
}

\usepackage{enumitem}%Itemise 

%Display Code, see also https://nasa.github.io/nasa-latex-docs/html/examples/listing.html
\usepackage{listings}
\definecolor{codeGreen}{RGB}{69, 161, 18}
\definecolor{backgroundcolor}{RGB}{255,255,255}
    \lstdefinestyle{mystyle}{
        language = Java,                            % Program language
        backgroundcolor=\color{backgroundcolor},    % Background color
        commentstyle=\color{blue},                  % Comment color
        keywordstyle=\color{codeGreen},             % Keyword style (see also bottom)
        numberstyle=\tiny\color{black},             % Line number color
        stringstyle=\color{olive},                  % String color
        basicstyle=\ttfamily\footnotesize,          % Fontsize
        breakatwhitespace=false,                    % Automatic breaks only at whitespace
        breaklines=true,                            % Automatic line breaking
        captionpos=b,                               % Caption position (here at bottom)
        keepspaces=true,                            % Keep spaces in text, i.e. for indentation
        numbers=none,                               % Line numbers (none, left, right)
        numbersep=5pt,                              % Separation from code
        showspaces=false,                           % Show spaces with underscores, overrides showstringspaces
        showstringspaces=false,                     % Underline spaces within strings only
        showtabs=false,                             % Underline tabs within strings only
        tabsize=2,                                  % Default tabsize in code
        morekeywords  = {final, override, String}   % Add keywords to set
    }\lstset{style=mystyle}                         % Set this as default

% color for  Titel / sub Titel
\definecolor{sectionbarcolor}{RGB}{148,0,255}
\definecolor{subsectionbarcolor}{RGB}{244, 38, 255}
\definecolor{subsubsectionbarcolor}{RGB}{255, 166, 234}
\definecolor{sectiontextcolor}{RGB}{255,255,255}
\definecolor{subsectiontextcolor}{RGB}{0,0,0}
\definecolor{subsubsectiontextcolor}{RGB}{0, 0, 0}


% Farben definieren, in "Schnittstellen"
\definecolor{purplecode}{RGB}{128, 0, 128} % Lila für Vehicle
\definecolor{turquoisecode}{RGB}{0, 128, 128} % Türkis für House


%section color box
\setkomafont{section}{\mysection}
\newcommand{\mysection}[1]{%
    \Large%
    \begin{tcolorbox}[colback=sectionbarcolor, coltext=sectiontextcolor, beforeafter skip=-0.15cm, boxrule=0pt, arc=2pt, left=0pt, right=0pt, top=0pt, bottom=0pt]%
        {#1}%
    \end{tcolorbox}%
}

%subsection color box
\setkomafont{subsection}{\mysubsection}
\newcommand{\mysubsection}[1]{%
    \Large%
    \begin{tcolorbox}[colback=subsectionbarcolor, coltext=subsectiontextcolor, beforeafter skip=-0.05cm, boxrule=0pt, arc=2pt, left=0pt, right=0pt, top=0pt, bottom=0pt]%
        {#1}%
    \end{tcolorbox}%
}

%subsubsection color box
\setkomafont{subsubsection}{\mysubsubsection}
\newcommand{\mysubsubsection}[1]{%
    \begin{tcolorbox}[colback=subsubsectionbarcolor, coltext=subsubsectiontextcolor, beforeafter skip=-0.05cm, boxrule=0pt, arc=2pt, left=0pt, right=0pt, top=0pt, bottom=0pt]%
        {#1}%
    \end{tcolorbox}%
}

%Information for maketitle
\title{\vspace{-1cm}Java für CPP-Programmierer}
\subtitle{HS 2024, Peter Bühler}
\author{Fabian Suter, Martina Knobel}
\date{{\small \today $\qquad$ 1.0.2}}

%Header & footer
\fancyhf{}
\setlength{\footskip}{0.5cm}
\fancyfoot[L]{\thepage{} / \pageref{LastPage}}
\fancyfoot[R]{Java}
\renewcommand{\footrulewidth}{0pt}
\renewcommand{\headrulewidth}{0pt}

\begin{document}
	\begin{multicols*}{3}
        \raggedcolumns
        \begin{minipage}{0.75\columnwidth}
		      \maketitle
        \end{minipage}
        \begin{minipage}{0.1\columnwidth}
            \includegraphics[width=0.7\linewidth]{pictures/Java-Logo.png}
        \end{minipage}
        \begin{minipage}{0.1\columnwidth}
            \begin{center}
                \quad
                \qrcode[height=1cm]{https://github.com/FabianSuter}
                \qquad    
            \end{center}
        \end{minipage}
        
        \thispagestyle{fancy}%Pagenumber for first page

        \section{Basics}

\subsection{Grundsätzlich}

Diese Zusammenfassung dient zur Hilfe beim Programmieren mit Java anhand von C++-Vorwissen. Es gibt einige wichtige Unterschiede, \textbf{Java:}

\begin{itemize}[itemsep=1pt, parsep=0pt]
    \item kennt keine Zeiger / Pointer
    \item kennt keine Funktionen (reine OO-Sprache)
    \item kennt kein Überladen von Operatoren
    \item kennt keine Destruktoren (Garbage-Collector gibt Speicher frei)
    \item kennt keine Mehrfachvererbung
    \item stellt eine umfangreiche Klassenbibliothek zur Verfügung
    \item -Programme sind plattformunabhängig
    \item ist weniger hardwarenah wie C++
    \item ...
\end{itemize}

\subsection{Sourcecode Hello world Java}

\lstinputlisting{code/HelloWorld.java}

% TODO: In Schleifen aufführen
% \subsection{For each Loop}\label{forEach}

% \verb|For each| - Loops laufen automatisch ein Array ab, ohne das zuerst die Grösse ermittelt werden muss. 
% Die C - \verb|for| - Loop Syntax wurde etwas erweitert.
% Syntax:

% \lstinputlisting{code/forEach.cpp}

% Siehe: Als Arrayname wird ein \say{Pointer} auf das Array verlangt, nicht auf das erste Element!  
% Ein \verb|For each| funktioniert \textbf{nicht} mit einem \textbf{Pointerpointer}. 


\subsection{Java Runtime}
Der Compiler erzeugt anders als in C++ \textbf{Bytecode}, welcher anschliessend auf einer Java Virtual MAchine (JVM) laufen kann.
Dadurch wird Java plattformunabhängig, da jede Plattform eine JVM hat, sei es Windows, Android oder MacOS.


\section{Datentypen}

Ein Datentyp besteht aus \textit{Werten} und \textit{Operationen}. Java besitzt zwei generelle Datentypen, 
namentlich \textbf{Primitive Datentypen} und \textbf{Referenzdatentypen}.

\subsection{Primitive Datentypen}
Sie haben plattformunabhängig den gleichen Wertebereich, es gibt keine \verb|unsigned|-Typen. Für bool'sche Werte gibt es den Datentyp \verb|boolean|.

\begin{center}
    \begin{tabular}{lll}
        \rowcolor[RGB]{239,239,239} 
        \textbf{Typ} & \textbf{Beschr.}        & \textbf{Beispiele} \\ \hline
        boolean      & Bool'scher Wert         & true, false \\
        char         & Textzeichen (UTF16)     & 'a', 'B', etc. \\
        byte         & Ganzzahl (8 Bit)        & -128 bis 127 \\
        short        & Ganzzahl (16 Bit)       & -32'768 bis 32'767 \\
        int          & Ganzzahl (32 Bit)       & $ -2^{31} $ bis $ 2^{31}-1 $ \\
        long         & Ganzzahl (64 Bit)       & $ -2^{63} $ bis $ 2^{63}-1 $, 1L \\
        float        & Gleitkommazahl (32 Bit) & 0.1f, 2e4f \\
        double       & Gleitkommazahl (64 Bit) & 0.1, 2e4 \\
    \end{tabular}
\end{center}

Gleitkommazahlen ohne Angaben sind automatisch \verb|double|.

\subsubsection{Überlauf / Unterlauf}
Bei \textbf{Ganzzahlen} ist der Überlauf in Java definiert, im Gegensatz zu C++.\\
2147483647 + 1 $ \rightarrow $ -2147483648\\

Bei \textbf{Gleitkommazahlen} gilt dasselbe:\\
2 * 1e308 $ \rightarrow $ \verb|POSITIVE_INFINITY|\\
5e-324 / 2 $ \rightarrow $ 0.0

\subsubsection{Undefinierte Operationen}
\textbf{Ganzzahlen} werfen bei Division/Modulo durch 0 einen Fehler bzw. eine Exception.\\

\textbf{Gleitkommazahlen} werfen bei Division durch 0 ein \verb|POSITIVE_INFINITY| resp. \verb|NEGATIVE_INFINITY|\\
Bei undefinierten Rechnungen wie 0 / 0 wird \verb|NaN| zurückgegeben.

\subsubsection{Text-Literale}
\textbf{char} mit Apostrophen: \verb|'A'|, \verb|'\n'| (NewLine), \verb|'\''| (Apostroph) \\

\textbf{String} mit Anführungszeichen: \verb|"Say \"hello\"!\n"|

\subsection{Referenzdatentypen}

\subsubsection{Arrays}
Arrays speichern wie in C++ mehrere Elemente mit selbem Datentyp. Der Zugriff erfolgt über Index, die Elemente liegen nebeneinander im Speicher.
Die Grösse des Arrays muss bei der Deklaration festgelegt werden und ist später nicht mehr änderbar.\\
Die Anzahl der Elemente kann in Java direkt mit \verb|length| abgerufen werden, siehe auch \textbf{\ref{Array-Loop}}.\\
Falls der Index ungültig ist, wird eine\\
\textbf{ArrayIndexOutOfBoundsException} geworfen.

\subsection{Typumwandlung}

\begin{minipage}{0.4\columnwidth}
    \textbf{Implizit:}\\
    Klein zu Gross,\\
    kein Cast-Operator erf.\\
    \lstinputlisting{code/implizit.java}
\end{minipage}
\begin{minipage}{0.55\columnwidth}
    \textbf{Explizit:}\\
    Gross zu Klein,\\
    Cast-Operator erf.\\
    \lstinputlisting{code/explizit.java}
\end{minipage}


        \section{Objektorientierung}

\subsection{Klassen und Objekte}{\label{Klassen}}
Klassen bestehen aus \textbf{Variablen} und \textbf{Methoden}.\\
Objekte können aus Klassen erzeugt werden: \verb|Point a = new Point()|\\
Konstruktoren können in den meisten IDEs automatisch generiert werden.

Sofern kein Konstruktor definiert ist, wird der \textbf{Default-Konstruktor} verwendet. Die Instanzvariablen werden mit
Default-Werten initialisiert (primitive mit \verb|0|, Referenzdatentypen mit \verb|NULL|)\\

Klassen arbeiten mit Referenzen, dies muss bei Zuweisungen beachtet werden:
\begin{center}
    \includegraphics[width=0.9\columnwidth]{pictures/copy-semantics.png}    
\end{center}

\subsection{Methoden}

\subsubsection{Parameter}
Java verwendet \textit{immer} \textbf{Call by Value}. Die Argumente werden kopiert und als Parameter übergeben. Dies kann zu unterschiedlichem Verhalten
je nach Datentyp führen:\\

\textbf{Primitive Datentypen}:\\
Methode arbeitet mit Kopie des Wertes
\begin{center}
    \includegraphics[width=0.9\columnwidth]{pictures/primitive-params.png}    
\end{center}

\textbf{Referenzdatentypen}:\\
Methode arbeitet mit Kopie der Referenz
\begin{center}
    \includegraphics[width=0.9\columnwidth]{pictures/reference-params.png}    
\end{center}

\subsubsection{Variadische Methoden}
Falls bei Funktionen die Anzahl der Argumente nicht im Voraus bekannt ist, kann folgende
Syntax verwendet werden: \verb|static int sum(int... numbers){}|\\
Der Compiler generiert ein Array aus der Parameterliste.

\subsection{Methodenreferenzen}
Anstatt Hilfsklassen für häufig verwendete Methoden zu erstellen, kann auch mit 
Methodenreferenzen gearbeitet werden, wie in C++ mit Funktionszeigern. Für die Referenzierung braucht es
eine Funktionsschnittstelle und die Implementierung. Methodenreferenzen benötigen eine aufrufende \textbf{höherwertige} Funktion. \\
Referenz-Varianten:\\
\begin{tabular}{ll}
    \verb|this::compare|    & Methode \verb|compare| im selben Objekt \\
    \verb|other::compare|   & Methode \verb|compare| in Objekt \textit{other} \\
    \verb|MyClass::compare| & Statische Methode in \textit{MyClass} \\
    \verb|MyClass::new|     & Konstruktor der Klasse \textit{MyClass} \\
\end{tabular}

\includegraphics[width=\linewidth]{pictures/methoden-referenzen.jpg}


\subsubsection{Lambdas}
Ad-Hoc Implementierung anstatt einer expliziten Methodenreferenz. \\
\includegraphics[width=\linewidth]{pictures/lambda.jpg}
\verb|Person|,\verb|{}| und \verb|return| sind optional, wenn nur ein Ausdruck enthalten ist

\textbf{Faustregel}: Lambdas sind kurz ($\sim$ 3 Zeilen), für längere Methodenreferenzen verwenden, siehe auch \ref{Lambdas}

\subsection{Unit Testing}
Unit Testing ist eine der Varianten, um Bugs zu verhindern. In guten Unit Tests sollen möglichst alle relevanten Fälle abgedeckt sein.
Dazu gehören Standardfälle (im Bereich der Funktion) und Edge Cases (z.B. 0, max. und min. Bereich, usw.). Siehe auch \ref{Unit-Test}\\
Faustregel:
\begin{itemize}
    \itemsep0em
    \item Pro Klasse eine Test-Klasse
    \item Pro Testfall eine Methode
\end{itemize}

\subsubsection{Assert-Methoden}
\begin{center}
    \begin{tabular}{ll}
        \rowcolor[RGB]{239,239,239} 
        \textbf{Methode} & \textbf{Bedingung} \\ \hline
        assertEquals(expected, actual) & für prim. und Referenztypen \\
        assertNotEquals(expected, actual) & \\
        assertSame(expected, actual) & actual == expected\\
        assertNotSame(expected, actual) & actual != expected\\
        assertTrue(condition) & condition\\
        assertFalse(condition) & !condition\\
        assertNull(value) & value == null\\
        assertNotNull(value) & value != null\\
    \end{tabular}
\end{center}

\begin{center}
    \includegraphics[width=0.9\columnwidth]{pictures/testmethode.png}    
\end{center}

\subsubsection{Sichtbarkeit}
\begin{center}
    \begin{tabular}{ll}
        \rowcolor[RGB]{239,239,239} 
        \textbf{Keyword} & \textbf{Sichtbar für} \\ \hline
        public & Alle Klassen \\
        protected & Klassen im selben Package und abg. Klassen\\
        (keines) & Klassen im selben Package \\
        private & Nur eigene Klasse \\
    \end{tabular}
\end{center}

Für Zugriffe auf \verb|private|-Variablen können Getter- und Setter-Methoden definiert werden. Siehe auch \ref{GetSet}


\subsection{Generics}

\textit{Disclaimer, folgende Themen sind hier nicht enthalten: Typebounds, Wildcardtypen, Covarianz, Type-Erasure}

Generics dienen dazu, typ-unabhängige Frames zu erstellen. 

\subsubsection{Generische Variablen}
Die Variable dient als Platzhalter innerhalb einer generischen Klasse, Methode oder Interface. Sie wird mit \verb|<>| umschlossen 
und kann innerhalb der Klasse,... wie ein normaler Typ verwendet werden.\\
Häufig verwendete Namen:\\
\begin{tabular}{l l|l l}\hline
    E & Element & T & Type\\
    K & Key     & V & Value\\
    N & Number  & S, U ,V,... & 2ter, 3ter, 4ter Type \\\hline
\end{tabular}\\
\\
Mehrere Typ-Variablen gleichzeitig sind zulässig, z.B. \verb|<T, U>|

\subsubsection{Generische Klassen}
Eine Klasse mit Typ-Variable. Die Variable dient als Platzhalter für unbekannten Typ. Verschachtelung bei der Anwendung ist zulässig.\\
\includegraphics[width=\linewidth]{pictures/generic-class.jpg}\\
Bei der Anwendung macht der Compiler eine statische Prüfung des Datentyps. der Type-Cast entfällt (auto-boxing, -unboxing)


\subsubsection{Generische Interfaces}\label{StackIterator}
Gleiche Syntaxregeln wie bei generischen Klassen, jede Klasse kann generische Interfaces implementieren.\\
Anwendung:\\
\includegraphics[width=\linewidth]{pictures/generic-interface1.jpg}
\hrule
\includegraphics[width=\linewidth]{pictures/generic-interface2.jpg}

\subsubsection{Generische Methoden}
Generische Methoden sind unabhängig von generischen Klassen und Interfaces. Die Implementation kann in normalen, 
nicht generischen Klassen erfolgen und ist auch in statischen Methoden zulässig.\\
Die Typ-Variable muss in Klammer \verb|<>| vor dem Rückgabewert stehen. \\
\verb|public static <T> List<T> merge(List<T> a, List<T> b) { … }|


        \section{Collections}{\label{Collections}}
In Collections können nur Referenzdatentypen abgelegt werden. Beim Hinzufügen des Elements wird das Objekt selber \textbf{nicht} kopiert,
es wird nur eine Referenz abgelegt. Grundlegende Collections:
\begin{tabular}{l l}
    $\cdot$ \textbf{List} & Folge von Elementen \\
    $\cdot$ \textbf{Set}  & Menge von Elementen \\
    $\cdot$ \textbf{Map}  & Abbildung Schlüssel $\rightarrow$ Werte \\
\end{tabular}

\subsection{Asymptotisches Laufzeitverhalten}
\begin{tabularx}{\linewidth}{l l X} \hline
    \textbf{Laufzeit} & \textbf{Beschr.} & \textbf{Beispiele} \\ \hline
    O(1)        & Konstant      & Indexzugriff Array \\
    O(log(n))   & Logarithmisch & Binärsuche \\
    O(n)        & Linear        & Lineare Suche \\
    O(n*log(n)) & LogLinear     & Schnelle Sortierverfahren: QuickSort, MergeSort \\
    O(n$^2$)    & Quadr.        & Einfache Sortierverfahren: SelectionSort, InsertionSort \\
    O(n$^3$)    & Kubisch       & Matrizen-Multiplikation \\
\end{tabularx}

\includegraphics[width=\columnwidth]{pictures/laufzeit-collections.jpg}

\subsection{Wrapper-Objekt}
Um primitive Datentypen in Collections verwenden zu können, müssen sie verpackt (\textit{Wrapping}) werden. Dies geschieht
meist \textbf{implizit}, es muss nur beim Datentyp der Collection definiert werden.
\begin{center}
    \includegraphics[width=\columnwidth]{pictures/wrapper-klassen.png}
\end{center}

\subsection{ArrayList}
ArrayLists sind eine geordnete Folge von Elementen mit demselben Referenzdatentyp. Elemente können einfach 
hinzugefügt oder entfernt werden. Duplikate oder \verb|null|-Einträge sind möglich\\
Der Zugriff auf Elemente erfolgt über Index (0 ... size()-1). Die Liste verwendet intern ein Array zur Verwaltung der Elemente.
Zu Beginn enthält sie, sofern nicht anders definiert, 10 Elemente und wird bei Erreichen der Kapazität mit Faktor 1.5 mulitpliziert.

\begin{center}
    \includegraphics[width=0.9\columnwidth]{pictures/arrayList-bsp.png}
\end{center}

\begin{center}
    \includegraphics[width=0.9\columnwidth]{pictures/arrayList-api.png}
\end{center}

\subsubsection{ArrayList: Kosten}
\begin{tabular}{l l l} \hline
    \textbf{Operation} & \textbf{Methode} & \textbf{Effizienz} \\ \hline
    Index-Zugriff & get(), set() & \color{green!80!black}Sehr schnell (direkter Zugriff) \\
    Hinzufügen    & add()        & \color{red}Langsam (umkopieren) \\
                  &              & \color{green!80!black}Sehr schnell (ohne umkop.) \\
    Entfernen     & remove(int)  & \color{red}Langsam (umkopieren) \\
    Finden        & contains()   & \color{red}Langsam (durchsuchen) \\
\end{tabular}

\subsection{LinkedList}
Funktioniert ähnlich wie ArrayList. Die Implementierungerfolgt mit einer doppelt-verketteten (vor- und rückwärts) Liste.
Es erfolgt kein Umkopieren beim Einfügen und Löschen von Elementen.

\subsubsection{LinkedList: Kosten}
\begin{tabular}{l l l} \hline
    \textbf{Operation} & \textbf{Methode} & \textbf{Effizienz} \\ \hline
    Index-Zugriff & get(), set() & \color{red} Langsam (traversieren) \\
    Hinzufügen    & add()        & \color{green!80!black}Sehr schnell (Knoten einhängen)\\
    Entfernen     & remove(int)  & \color{red}Langsam in Mitte \\
                  &              & \color{green!80!black}Sehr schnell am Anfang und Ende \\
    Finden        & contains()   & \color{red}Langsam (traversieren) \\
\end{tabular}

\subsection{HashSet vs. TreeSet}
Sets sind Container für Mengen, wobei Duplikate \textbf{nicht} erlaubt sind. Die Gleichheit wird mit \verb|equals()| geprüft.\\
HashSets sind \textbf{unsortiert} und \textbf{oft sehr effizient}\\
\verb|Set<String> firstSet = new HashSet<>();|

\begin{minipage}{0.5\columnwidth}
    \includegraphics[width=0.9\linewidth]{pictures/hashset.jpg}
\end{minipage}
\hfill
\begin{minipage}{0.45\columnwidth}
    Elemente liefern \verb|hashCode()| konsistent zu \verb|equals()|
\end{minipage}

TreeSets sind \textbf{sortiert} und \textbf{immer effizient}\\
\verb|Set<String> firstSet = new TreeSet<>();|

\begin{minipage}{0.5\columnwidth}
    \includegraphics[width=0.9\linewidth]{pictures/treeset.jpg}
\end{minipage}
\hfill
\begin{minipage}{0.45\columnwidth}
    Elemente implementieren \verb|Comparable| und \verb|equals()|
\end{minipage}

\subsubsection{HashSet vs. TreeSet: Kosten}
\begin{tabular}{l l l} \hline
    \textbf{Operation} & \textbf{TreeSet} & \textbf{HashSet }\\ \hline
    Finden      & \color{yellow!75!red} Schnell    & \color{green!80!black}Sehr schnell \\
    Einfügen    & \color{yellow!75!red} Schnell    & \color{green!80!black}Sehr schnell \\
    Löschen     & \color{yellow!75!red} Schnell    & \color{green!80!black}Sehr schnell (nur bei ``guter'' Impl.) \\
    Sortierung  & \color{green!80!black} Ja & \color{red}Nein                    \\
\end{tabular}

\subsection{HashMap vs. TreeMap}
Maps sind für Mengen von Schlüssel-Wert-Paaren. Jedem Schlüssel ist genau ein Wert zugeordnet. Es sind \textbf{keine} doppelten Schlüssel erlaubt. \\
Beispiel:\\
\includegraphics[width=0.7\linewidth]{pictures/map-beispiel.jpg}

HashMaps sind \textbf{unsortiert} und \textbf{oft sehr effizient}\\
\verb|Map<Integer, Student> masters = new HashMap<>();|
\\
\\
TreeMaps sind \textbf{nach Schlüssel sortiert} und \textbf{immer effizient}\\
\verb|Map<Integer, Student> masters = new TreeMap<>();|

\subsubsection{HashMap vs. TreeMap: Kosten}
\begin{tabular}{l l l} \hline
    \textbf{Operation} & \textbf{TreeMap} & \textbf{HashMap }\\ \hline
    Finden      & \color{yellow!75!red} Schnell             & \color{green!80!black}Sehr schnell \\
    Einfügen    & \color{yellow!75!red} Schnell             & \color{green!80!black}Sehr schnell \\
    Löschen     & \color{yellow!75!red} Schnell             & \color{green!80!black}Sehr schnell \\
                &                                           & \color{green!80!black}(nur bei ``guter'' Impl.) \\
    Sortierung  & \color{green!80!black} Ja, nach Schlüssel & \color{red}Nein                    \\
\end{tabular}

\subsection{equals(), Hashing}
Der Operator \verb|==| liefert einen Referenzvergleich, die Methode \textbf{equals()} ist für den inhaltlichen Vergleich.\\
Alle Klassen erben equals() von \textit{Object}. Die Default-Implementation liefert \verb|a == b| (Referenzvergleich). \\
Bei einigen Klassen, z.B. \textit{String, Integer, ...} ist equals() bereits überschrieben.\\
Eine Klasse \textbf{muss} equals() von \textit{Object} überschreiben:\\
\includegraphics[width=\linewidth]{pictures/equals-impl.jpg}

\subsubsection{Hashing: Konzept}
Die Funktion \verb|hashCode()| bildet ein Objekt auf seinen Hash-Code ab, welcher den Ablegeort des Objekts definiert.\\
Gleiche Objekte können den gleichen Hashcode haben.\\
\textit{ACHTUNG : gleicher Hashcode muss aber nicht gleiches Objekt sein!}

\subsubsection{Hashfunktion}
Als Faustregel: Bei jedem \textbf{equals()} gleich ein \textbf{hashCode()} schreiben:\\
\includegraphics[width=\linewidth]{pictures/hashcode.jpg}
        \section{Vererbung}
Die Vererbung funktioniert in Java ähnlich wie in C++. Eine Klasse in Java kann jedoch nur \textbf{eine} Basisklasse haben.
Eine abgeleitete Klasse erbt die Instanzvariablen und Instanzmethoden der Basisklasse.
Die oberste Klasse aller Basisklassen ist \verb|Object|. Methoden von \verb|Object| sind:
\begin{itemize}
    \itemsep0em
    \item \verb|public String toString()|
    \item \verb|public boolean equals(Object obj)|
    \item \verb|public int hashCode()|
    \item ...
\end{itemize}

\subsection{Impliziter Code in Vererbung}
\begin{center}
    \includegraphics[width=0.9\columnwidth]{pictures/vererbung-implizit.png}
\end{center}

\subsection{Konstruktor bei Vererbung}
\begin{center}
    \includegraphics[width=0.9\columnwidth]{pictures/vererbung-konstr.png}
\end{center}

\subsection{Overriden von Methoden}
Gleiche Funktion wie das Keyword \verb|virtual| in C++, dieses gibt es jedoch nicht in Java. Stattdessen wird
vor einer neu implementierten Methode einer Subklasse das Schlüsselwort \verb|@Override| gesetzt. Dies ist optional, aber sinnvoll.\\
\verb|@Override|\\
\verb|public void print()|\\

Mit \verb|super| wird eine überschriebene Methode aufgerufen.
\begin{center}
    \includegraphics[width=0.9\columnwidth]{pictures/super.png}
\end{center}

\subsection{Abstrakte Klassen}{\label{AbstractClass}}
Schlüsselwort: \verb|abstract|\\

Eine abstrakte Klasse ist nicht vollständig implementiert, sprich einzelne Methoden können nicht implementiert 
sein und die Klasse kann nicht instanziiert werden.

Sie dient als Basistyp für Sub-Klassen (statischer Typ) und vererbt ihre Grundfunktionalität an Sub-Klassen.\\
Beispiel:\\
\begin{minipage}{0.5\columnwidth}
    \begin{center}
        \includegraphics[width=0.9\columnwidth]{pictures/abstrakte-Klasse-Bsp.png}
    \end{center}
\end{minipage}
\hfill
\begin{minipage}{0.5\columnwidth}
    \begin{center}
        \includegraphics[width=0.9\columnwidth]{pictures/abstrakte-Klasse-Bsp2.png}
    \end{center}
\end{minipage}

\section{Binding}
\subsection{Dynamic Binding}
Generell bei nicht-privaten Instanzmethoden

\subsection{Static Binding}
Generell bei privaten Instanzmethoden (In Subklasse nicht mehr sichtbar $\rightarrow$ Neudef. der Methode) und statischen Methoden



        {\small
\section{Schnittstellen}
    \begin{tabular}{l}
        $\bullet$ Dient als Schleuse zwischen Klasse und Aussenwelt.\\
        $\bullet$ Die Klasse muss die Funktionalität \textit{implementieren}\\
        $\bullet$ Die Aussenwelt darf die Funktionalität \textit{nutzen}.\\\hline
        $\bullet$ Methode(n) einer Schnittstelle: implizit \verb|public| und \verb|abstract|.\\
        $\bullet$ Deshalb werden nur \textbf{Methodendeklarationen} aufgeführt,\\
        $\qquad$ alles andere ist ungültig/unnötig.\\\hline

        $\bullet$ Diese Taktik wird als \textbf{lose Kopplung} bezeichnet.\\
        $\bullet$ Erlaubt unabhängige Entwicklung verschiedener Teams.\\\hline

        $\bullet$ Mehrere Klassen können eine Schnittstelle implementieren\\
        $\bullet$ Klasse kann aber auch mehrere Schnittstellen implementieren\\
        $\qquad$ $\rightarrow$ \textbf{Mehrfach-Implementierung} erlaubt.\\
    \end{tabular}
    \vspace{-0.3cm}

\subsection{Abstrakte Klassen vs. Interfaces}
    \begin{tabularx}{\linewidth}{|X c X|} \hline
        \tikz[baseline=(text.base)]\node[fill=blue, fill opacity=0.3, text opacity=1, rounded corners, inner sep=2pt, minimum height=5pt] (text) {\textbf{Abstrakte Klassen}}; & & \tikz[baseline=(text.base)]\node[fill=green, fill opacity=0.3, text opacity=1, rounded corners, inner sep=2pt, minimum height=5pt] (text) {\textbf{Interfaces}}; \\
        (siehe auch \ref{AbstractClass} ) enthalten Instanzvariablen, Konstruktoren und teilweise implementierte Methoden & $\longleftrightarrow $ & enthalten nur Deklarationen, keinen Code \\
        \hline
    \end{tabularx}

    \begin{tabular}{l}
        \rowcolor[RGB]{239,239,239} 
        \textbf{Wann \tikz[baseline=(text.base)]\node[fill=green, fill opacity=0.3, text opacity=1, rounded corners, inner sep=2pt, minimum height=5pt] (text) {Interfaces};?}\\\hline
        $\bullet$ Implementierung (noch) nicht bekannt\\
        $\bullet$ Implementierungen haben wenig gemeinsamen Code\\
        $\bullet$ Losere Kopplung\\\hline
        \rowcolor[RGB]{239,239,239} 
        \textbf{Wann \tikz[baseline=(text.base)]\node[fill=blue, fill opacity=0.3, text opacity=1, rounded corners, inner sep=2pt, minimum height=5pt] (text) {abstrakte Klassen};?}\\\hline
        $\bullet$ Code bei mehreren Klassen wiederverwenden\\
        $\bullet$ Klassen haben gemeinsame Instanzvar. und Methoden\\
        $\bullet$ Konstruktor erforderlich, um Instanzvar. zu init.\\
    \end{tabular}
    \vspace{-0.3cm}

\subsection{Ein Interface - mehrere Implementierung}
    \vspace{-0.2cm}
    \begin{center}
        \begin{minipage}{0.4\columnwidth}
                % Code für Vehicle
                \lstinputlisting{code/interfaceVehicle.java}
        \end{minipage}
    \end{center}
    \vspace{-0.6cm}
    \begin{center}
        \begin{minipage}{0.47\columnwidth}% Code für RegularCar (links)
            \lstset{
                basicstyle=\scriptsize\ttfamily\color{purplecode}, % Kleinere Schriftgröße + Violett Stil
                keywordstyle=\bfseries\color{purplecode},
                identifierstyle=\color{purplecode},
            }
            \lstinputlisting{code/RegularCar.java}
        \end{minipage}
        \hspace{0.01\columnwidth} % Abstand für den Strich
        \vrule width 0.5pt % Vertikaler Strich
        \hspace{0.01\columnwidth} % Abstand für den Strich
        \begin{minipage}{0.47\columnwidth} % Code für RacingCar (rechts)
            \lstset{
                basicstyle=\scriptsize\ttfamily\color{turquoisecode}, % Kleinere Schriftgröße + Türkis Stil
                keywordstyle=\bfseries\color{turquoisecode},
                identifierstyle=\color{turquoisecode},
            }
            \lstinputlisting{code/RacingCar.java}
        \end{minipage}
    \end{center}
    \vspace{-0.2cm}

\subsection{Mehrere Interfaces - eine Implementierung}
    \vspace{-0.2cm}
    \begin{center}
        % Code für Vehicle (links)
        \begin{minipage}{0.49\columnwidth}
            \lstset{
                basicstyle=\scriptsize\ttfamily\color{purplecode}, % Kleinere Schriftgröße + Violett Stil
                keywordstyle=\bfseries\color{purplecode},
                identifierstyle=\color{purplecode},
            }
            \lstinputlisting{code/interfaceVehicle.java}
        \end{minipage}
        \hfill
        % Code für House (rechts)
        \begin{minipage}{0.49\columnwidth}
            \lstset{
                basicstyle=\scriptsize\ttfamily\color{turquoisecode}, % Kleinere Schriftgröße + Türkis Stil
                keywordstyle=\bfseries\color{turquoisecode},
                identifierstyle=\color{turquoisecode},
            }
            \lstinputlisting{code/interfaceHouse.java}
        \end{minipage}
    \end{center}
    \vspace{-0.9cm}
    \begin{center}
        \begin{minipage}{0.8\columnwidth}
            % MobileHome
            {\scriptsize
            \lstinputlisting{code/mobileHome.java}
            \vspace{-0.6cm}
            \textcolor{purplecode}{\lstinputlisting{code/mobileHome_1.java}}
            \vspace{-0.6cm}
            \textcolor{turquoisecode}{\lstinputlisting{code/mobileHome_2.java}}}
        \end{minipage}
    \end{center}
    \vspace{-0.6cm}
}
        \section{Exceptions, Errors}
Errors:
\begin{itemize}
    \itemsep0em
    \item Schwerwiegende Fehler, \textbf{nicht behandeln!}
    \item Fehler in JVM: OutOfMemoryError, ...
    \item Programmierfehler: AssertionError
\end{itemize}
Exceptions:
\begin{itemize}
    \itemsep0em
    \item Laufzeitfehler, die \textbf{behandelbar} sind
    \item fehlerhafte Bedienung, Parameter, ...
    \item siehe auch Checked / Unchecked \ref{checked,unchecked}
\end{itemize}
% Exceptions (Ausnahmen) dienen als Mittel zur kontrollierten Reaktion von Laufzeitfehlern, 
% z.B. bei logischen Programmfehlern, fehlerhafter Bedienung oder Probleme der JVM.

\subsubsection{Checked, Unchecked}\label{checked,unchecked}
Checked:
\begin{itemize}
    \itemsep0em
    \item Exception muss behandelt werden ODER
    \item \verb|throws|-Deklaration im Methodenkopf
    \item Wird vom Compiler geprüft
\end{itemize}

Unchecked:
\begin{itemize}
    \itemsep0em
    \item Kein \verb|throws| oder Behandlung nötig
    \item \verb|RuntimeException| und Error, sowie ihre Unterklassen
    \item Wird nicht vom Compiler geprüft
\end{itemize}

\subsection{Exception auslösen}
\includegraphics[width=\linewidth]{pictures/exception-throw.jpg}

\subsubsection{throws-Deklaration}
Die Methode muss alle potentiellen Exceptions deklarieren, die der Aufrufer erhalten könnte. (Ausser Unchecked \ref{checked,unchecked})

Der Aufrufer muss die Exception behandeln (fangen) oder weiterreichen.

\subsection{Exception behandeln}
\begin{minipage}{0.6\columnwidth}
    \begin{itemize}
        \itemsep0em
        \item Regulärer Code (\verb|try-Block|)
        \item Fehlerbehandlung (\verb|catch|-Block)
        \begin{itemize}
            \itemsep0em
            \item Exception im \verb|try-Block| $\rightarrow$ \verb|catch|-Block
            \item Keine Exception: \verb|catch|-Block wird nicht ausgeführt
        \end{itemize}
        \item Opt. \verb|finally|-Block: Wird immer durchlaufen
    \end{itemize}
\end{minipage}
\hfill
\begin{minipage}{0.35\columnwidth}
    \includegraphics[width=\linewidth]{pictures/try-catch-finally.jpg}
\end{minipage}

Bei mehreren \verb|catch|-Blöcken wird \textbf{nur} der erste Passende (von oben nach unten gesucht) ausgeführt. 

Falls Exception nicht behandelt wird, wird die Exception zum nächsthöheren Aufrufer geschickt. 
Wenn dies \verb|main()| ist, wird die Exception an die JVM geworfen und das Programm abgebrochen.

\subsubsection{try-with-resources}
Für Objekte, die geschlossen werden müssen (Interface \verb|AutoCloseable|)\\
\includegraphics[width=\linewidth]{pictures/try-with-resources.jpg}

        {\small
\section{I/O-Streams $\qquad\qquad$ java.io.*}
    \begin{tabular}{ll}
        \rowcolor[RGB]{239,239,239} 
        \textbf{Input Stream} & \textbf{Output Stream}\\\hline
        Daten \textbf{von aussen} lesen & Daten \textbf{nach aussen} schreiben\\
        $\bullet$ Tastatur & $\bullet$ Bildschirm/Konsole\\
        $\bullet$ Netzwerk & $\bullet$ Netzwerk\\
        $\bullet$ Dateien & $\bullet$ Dateien\\
        \includegraphics[width=0.46\linewidth]{pictures/streams_1.jpg} & \includegraphics[width=0.46\linewidth]{pictures/streams_2.jpg}\\
    \end{tabular}
    \vspace{-0.2cm}
    \begin{tabular}{ll}
        \rowcolor[RGB]{239,239,239} 
        \textbf{Byte-Streams}            & \textbf{Character-Streams}\\
        $\bullet$ 8-Bit-Daten            & $\bullet$ 16-Bit Textzeichen (UTF-16)\\
        $\bullet$ Klassen erben von      & $\bullet$ Klassen erben von \verb|Reader, Writer|\\
        \verb|InputStream, OutputStream| & $\bullet$ Zeichen- / Zeilenweise Ein- \& Ausgabe\\
    \end{tabular}
    \vspace{-0.1cm}

\subsection{Byte-Streams}
    \begin{tabular}{l}
        \textbf{InputStream}: \verb|int read(byte[] b, int offset, int length)|\\
        \textbf{OutputStream}: \verb|void write(byte[] b, int offset, int length)|\\
        Lese/schreibe \verb|length| Bytes in Array \verb|b| ab Index \verb|offset|\\\hline
        \verb|void flush()|: Implizit bei \verb|close()|\\
    \end{tabular}
    \vspace{-0.3cm}

    \subsubsection{Standard Input/Output}
        \begin{tabular}{l}
            \verb|System.in| $\rightarrow$  \textit{InputStream}\\\hline
            \verb|System.out|, \verb|System.err| $\rightarrow$ \textit{PrintStream} {\footnotesize(Subklasse von \textit{OutputStream})}\\
        \end{tabular}
        \vspace{-0.3cm}

    \subsubsection{FileInput: Ganze Datei binär einlesen (kann speicherintensiv werden)}
        \verb|byte[] data = Files.readAllBytes(Path.of("in.bin"));|
        \hrule
        \includegraphics[width=0.85\linewidth]{pictures/file-in.jpg}
        \vspace{-0.2cm}

    \subsubsection{FileOutput: Ganze Datei binär schreiben}
        \verb|Files.write(Path.of("out.bin"), data);|
        \hrule
        \includegraphics[width=0.85\linewidth]{pictures/file-out.jpg}

        \verb|new FileOutputStream("test.data", true)| um an existierende Datei anzuhängen
        \vspace{-0.2cm}

\subsection{Character-Stream}
    \subsubsection{FileReader}
        \includegraphics[width=0.9\linewidth]{pictures/filereader.jpg}\\
        \verb|new FileReader(f)| \\
        $\updownarrow$ \\
        \verb|new InputStreamReader(new FileInputStream(f))|
        \vspace{-0.3cm}

    \subsubsection{FileWriter}
        \includegraphics[width=0.8\linewidth]{pictures/filewriter.jpg}
        \vspace{-0.3cm}

    \subsubsection{Zeilenweises Lesen}
        \includegraphics[width=0.8\linewidth]{pictures/zeilenweise.jpg}
        \vspace{-0.3cm}

    \subsubsection{Einfachster Text-Datei-Zugriff}
        \begin{tabular}{l}
            Ganze Text-Datei lesen \\
            \verb|List<String> lines = Files|\verb|.readAllLines(Path.of("in.txt"),| \\
            \verb|StandardCharsets.UTF_8);| \\
            \\
            Ganze Text-Datei schreiben \\
            \verb|Files.write(Path.of("out.txt"),| \\
            \verb|lines, StandardCharsets.UTF_8);| \\
        \end{tabular}
        \vspace{-0.3cm}

\subsection{Serialisierung}
    \begin{tabular}{l}
        \verb|Serializable|-Interface implentieren\\
        \verb|class Person implements Serializable {..}|\\
        $\rightarrow$ Klassen in Files einfach abzuspeichern \& wieder reinzuladen\\
        \tikz[baseline=(text.base)]\node[fill=red, fill opacity=0.2, text opacity=1, rounded corners, inner sep=2pt, minimum height=5pt] (text) {ACHTUNG:}; Wird die Klasse vor dem Deserialisieren abgeändert,\\
        $\qquad$ z.B. eine neue Variable, funktioniert dies nicht!\\
    \end{tabular}
    \lstinputlisting{code/serializing.java} 

}\vspace{-0.3cm}
        \section{Threads}
Schnellere Programme:
\begin{itemize}[itemsep=0em, parsep=0pt]
    \item Aufteilen einer Datenmenge in mehrere Teile, um parallel zu arbeiten
    \item Teilresultate nach Verarbeitung zusammenführen
    \item Bsp.: Sortierverfahren, Komprimierung, Bildverarbeitung
\end{itemize}
Einfachere Programme:
\begin{itemize}[itemsep=0em, parsep=0pt]
    \item Gleichzeitig oder verzahnt ausführbare Abläufe
    \item Bsp.: Layout, Speichern im Hintergrund
\end{itemize}
Multi-Core-Prozessoren können mehrere Threads parallel ausführen, für jeden Core zwei Threads.

\subsection{JVM Thread Modell}
Java ist ein Single Process System. JVM erzeugt beim Aufstarten einen Thread, welcher \verb|main()| aufruft. Der Programmierer, Subsysteme oder das 
Laufzeitsystem können ebenfalls Threads starten.

Die JVM läuft, solange Threads laufen, ausser wenn Threads als \textit{Daemon} markiert sind (z.B. Garabge Collector). JVM wartet nicht auf Daemon Threads, 
diese werden bei JVM-Ende unkontrollliert abgebrochen.

\subsubsection{Runnable Interface}

\begin{minipage}{0.48\columnwidth}
    \includegraphics[width=\linewidth]{pictures/runnable-interface.jpg}
\end{minipage}
\hfill
\begin{minipage}{0.48\columnwidth}
    Kann auch als Lambda übergeben werden:\\
    \includegraphics[width=\linewidth]{pictures/runnable-lambda.jpg}
\end{minipage}
Funktionsschnittstelle eines Threads, sie wird beim Starten des Threads durch die JVM gerufen.

\subsubsection{Start und Ende}
Ein Thread wird nach \verb|start()| ausgeführt (über \verb|run()| des \verb|Runnable|-Interfaces):\\
\lstinputlisting{code/thread.java}

Der Thread endet beim Verlassen von \verb|run|, z.B. durch Ende der MEthode, Return Statemnt oder unbeh. Exception

\subsection{Multi-Thread Beispiel}
\lstinputlisting{code/Demo02MultiThread.java}

\subsection{Alternative Implementationen}
\subsubsection{Explizit}
\lstinputlisting{code/MyRunnable.java}

\subsubsection{Sub-Klasse von Thread}
\lstinputlisting{code/SimpleThread.java}

% \subsection{Ablaufdefinitionen}
% \subsubsection{join()}
% \includegraphics[width=\linewidth]{pictures/thread-join.jpg}

\subsection{Thread-Methoden}

\subsubsection{Thread Passivierung}
\paragraph{Thread.sleep(milliseconds)}
Laufender Thread wird schlafen gelegt

\paragraph{Thread.yield()}
Laufender Thread gibt Prozessor frei und wird wieder ablaufbereit

\subsubsection{InterruptedException}
Mögliche Exception bei blockierenden Aufrufen, z.B. \verb|join(), sleep(), ...|

Threads können von aussen unterbrochen werden: \verb|myThread.interrupt()| $\rightarrow$ bricht \verb|join(), sleep(), ...| ab

\subsubsection{Weitere Methoden}
\verb|static Thread currentThread()| liefert Instanz des Threads \\

\verb|long threadId()| liefert ID des Threads \\

\verb|void setDaemon(boolean on)| Thread als \textit{Daemon} markieren

\subsection{Synchronisation}
Threads teilen sich Adressraum und Heap. Wird auf dasselbe Objekt zugegriffen, muss abgesichert werden.

\verb|synchronized| ist ein Modifier für Methoden, ähnlich wie ein Flag.
\lstinputlisting{code/BankAccount.java}
Nur ein Thread kann eine der \verb|synchronized|-Methoden zur gleichen Zeit in derselben Instanz ausführen
        {\small
\section{Stream-API $\qquad\qquad$ java.util.stream.*}
    \begin{tabular}{l}
        $\bullet$ Wird für deklarative Abfragen von Collections gebraucht\\
        $\bullet$ Code definiert, \textbf{was} gesucht wird, nicht \textbf{wie}\\
        $\bullet$ Framework arbeitet sehr intensiv mit Lambdas\\
        $\bullet$ Ist komplett unabhängig von Input/Output-Streams\\    
    \end{tabular}

    \begin{tabular}{l l}
        \rowcolor[RGB]{239,239,239} 
        \textbf{Basisschnittstellen} $\qquad$ & \textbf{Für primitive Datentypen}\\
        $\bullet$ \verb|Stream<T>| &$\bullet$ \verb|IntStream|\\
        & $\bullet$ \verb|LongStream|\\
        & $\bullet$ \verb|DoubleStream|\\
    \end{tabular}
    \vspace{-0.3cm}

\subsection{Endliche Quellen}
    \begin{tabular}{l l}
        \verb|list.stream()              | & Liefert Stream anh. Collection \\
        \verb|Arrays.stream(array)       | & Liefert Stream anh. Array \\
        \verb|IntStream.range(1, 100)    | & Zahlen von 1 bis 100 \\
        \verb|Stream.of(2, 3, 5, 7)      | & eigene Aufzählung \\
        \verb|Stream.concat(strm1, strm2)| & Verketteter Stream \\
    \end{tabular}
    \vspace{-0.3cm}

\subsection{Unendliche Quellen}
    \begin{minipage}{0.5\linewidth}
        \verb|generate()| \\
        \includegraphics[width=1.2\linewidth]{pictures/generate.jpg}
    \end{minipage}
    \hfill
    \begin{minipage}{0.5\linewidth}
        \verb|iterate()| \\
        \includegraphics[width=1.2\linewidth]{pictures/iterate.jpg}
    \end{minipage}
    \vspace{-0.3cm}

\subsection{Zwischenoperationen}
    \begin{tabular}{l l}
        \verb|filter(Predicate) | & Filtern mit Lambda \\
        \verb|map(Function)     | & Mappen mit Lambda \\
        \verb|mapToInt(Function)| & Mappen mit \verb|int,long, double| \\
        \verb|sorted(Comparator)| & Sortieren mit Comparator \\
        \verb|distinct()        | & Duplikate entfernen gemäss \verb|equals()| \\
        \verb|limit(long n)     | & n-Elemente liefern \\
        \verb|skip(long n)      | & n-Elemente überspringen \\
    \end{tabular}

    Zwischenoperationen dürfen die Collection \textbf{nicht ändern} und sie dürfen keine Abhängigkeit zu äusseren, änderbaren Variablen haben.
    \vspace{-0.3cm}

\subsection{Terminaloperationen (beenden Stream)}
    \begin{tabular}{l l}
        \verb|forEach(Consumer)     | & Pro Element Operation anwenden \\
        \verb|count()               | & Anzahl Elemente \\
        \verb|min(),max()           | & bei \verb|Stream<T>| Comparator-Arg. erf. \\
        \verb|average(), sum()      | & Nur bei in/long/double-Stream \\
        \verb|findAny(), findFirst()| & Gibt irgendein/erstes Element zurück \\
        \verb|collect()|              & Rückumwandlung zu Collection \\
        \verb|toArray()|              & Rückumwandlung zu Array \\
    \end{tabular}
    \vspace{-0.3cm}

    \subsubsection{Collectors $\rightarrow$ collect(...)}
        \verb|Collectors.toList()| $\rightarrow$ in Liste abbilden

        \verb|Collectors.toCollection(TreeSet::new)| \\
        $\hookrightarrow$ in beliebige Collection abbilden (Konstruktorreferenz)

        \verb|Collectors.groupingBy(key, aggregator)|:

        \begin{tabular}{l}
            $\bullet$ Gruppierung mit opt. Aggregator\\
            $\bullet$ Aggregator: averaging, summing, counting\\
            $\bullet$ Liefert HashMap als Rückgabewert\\
        \end{tabular}
        \vspace{-0.3cm}

\subsection{Funktionsschnittstellen}
    \subsubsection{Vordefinierte}
        \includegraphics[width=0.8\linewidth]{pictures/funktionsschnitt-vordef_1.jpg}
        \includegraphics[width=0.8\linewidth]{pictures/funktionsschnitt-vordef_2.jpg}
        \includegraphics[width=0.8\linewidth]{pictures/funktionsschnitt-vordef_3.jpg}
        \vspace{-0.3cm}

    \subsubsection{Optional-Wrapper}
        \verb|average(), min(), max(), findAny(), findFirst()| \\
        \includegraphics[width=0.8\linewidth]{pictures/optional-wrapper.jpg}
        \vspace{-0.3cm}

    \subsubsection{Matching}
        \verb|allMatch(), anyMatch(), noneMatch()| \\
        Prüfen, ob das Prädikat auf alle / irgendein/ kein Element zutrifft \\
        \verb|boolean 18plus = ppl.stream().allMatch(p->p.getAge >= 18);|
        \vspace{-0.2cm}

}
        \section{Code-Snippets}

\subsection{Fakultät}
    \lstinputlisting{code/Factorial.java}

\subsection{Array-Loop}{\label{Array-Loop}}
    \lstinputlisting{code/arrays.java}

\subsection{Schleife mit continue}
    \lstinputlisting{code/continue.java}

\subsection{Unit Test}{\label{Unit-Test}}
    \lstinputlisting{code/unitTest.java}

\subsection{Getter- und Settermethoden}{\label{GetSet}}
    \lstinputlisting{code/gettersetter.java}

\subsection{Concatenate Strings}
    \lstinputlisting{code/VariadicMethodString.java}

        
	\end{multicols*}
\end{document}
