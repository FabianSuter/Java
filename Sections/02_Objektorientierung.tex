\section{Objektorientierung}

\subsection{Klassen und Objekte}{\label{Klassen}}
Klassen bestehen aus \textbf{Variablen} und \textbf{Methoden}.\\
Objekte können aus Klassen erzeugt werden: \verb|Point a = new Point()|\\
Konstruktoren können in den meisten IDEs automatisch generiert werden.

Sofern kein Konstruktor definiert ist, wird der \textbf{Default-Konstruktor} verwendet. Die Instanzvariablen werden mit
Default-Werten initialisiert (primitive mit \verb|0|, Referenzdatentypen mit \verb|NULL|)\\

Klassen arbeiten mit Referenzen, dies muss bei Zuweisungen beachtet werden:
\begin{center}
    \includegraphics[width=0.9\columnwidth]{pictures/copy-semantics.png}    
\end{center}

\subsection{Methoden}

\subsubsection{Parameter}
Java verwendet \textit{immer} \textbf{Call by Value}. Die Argumente werden kopiert und als Parameter übergeben. Dies kann zu unterschiedlichem Verhalten
je nach Datentyp führen:\\

\textbf{Primitive Datentypen}:\\
Methode arbeitet mit Kopie des Wertes
\begin{center}
    \includegraphics[width=0.9\columnwidth]{pictures/primitive-params.png}    
\end{center}

\textbf{Referenzdatentypen}:\\
Methode arbeitet mit Kopie der Referenz
\begin{center}
    \includegraphics[width=0.9\columnwidth]{pictures/reference-params.png}    
\end{center}

\subsubsection{Variadische Methoden}
Falls bei Funktionen die Anzahl der Argumente nicht im Voraus bekannt ist, kann folgende
Syntax verwendet werden: \verb|static int sum(int... numbers){}|\\
Der Compiler generiert ein Array aus der Parameterliste.

\subsection{Methodenreferenzen}
Anstatt Hilfsklassen für häufig verwendete Methoden zu erstellen, kann auch mit 
Methodenreferenzen gearbeitet werden, wie in C++ mit Funktionszeigern. Für die Referenzierung braucht es
eine Funktionsschnittstelle und die Implementierung. Methodenreferenzen benötigen eine aufrufende \textbf{höherwertige} Funktion. \\
Referenz-Varianten:\\
\begin{tabular}{ll}
    \verb|this::compare|    & Methode \verb|compare| im selben Objekt \\
    \verb|other::compare|   & Methode \verb|compare| in Objekt \textit{other} \\
    \verb|MyClass::compare| & Statische Methode in \textit{MyClass} \\
    \verb|MyClass::new|     & Konstruktor der Klasse \textit{MyClass} \\
\end{tabular}

\includegraphics[width=\linewidth]{pictures/methoden-referenzen.jpg}


\subsubsection{Lambdas}
Ad-Hoc Implementierung anstatt einer expliziten Methodenreferenz. \\
\includegraphics[width=\linewidth]{pictures/lambda.jpg}
\verb|Person|,\verb|{}| und \verb|return| sind optional, wenn nur ein Ausdruck enthalten ist

\textbf{Faustregel}: Lambdas sind kurz ($\sim$ 3 Zeilen), für längere Methodenreferenzen verwenden, siehe auch \ref{Lambdas}

\subsection{Unit Testing}
Unit Testing ist eine der Varianten, um Bugs zu verhindern. In guten Unit Tests sollen möglichst alle relevanten Fälle abgedeckt sein.
Dazu gehören Standardfälle (im Bereich der Funktion) und Edge Cases (z.B. 0, max. und min. Bereich, usw.). Siehe auch \ref{Unit-Test}\\
Faustregel:
\begin{itemize}
    \itemsep0em
    \item Pro Klasse eine Test-Klasse
    \item Pro Testfall eine Methode
\end{itemize}

\subsubsection{Assert-Methoden}
\begin{center}
    \begin{tabular}{ll}
        \rowcolor[RGB]{239,239,239} 
        \textbf{Methode} & \textbf{Bedingung} \\ \hline
        assertEquals(expected, actual) & für prim. und Referenztypen \\
        assertNotEquals(expected, actual) & \\
        assertSame(expected, actual) & actual == expected\\
        assertNotSame(expected, actual) & actual != expected\\
        assertTrue(condition) & condition\\
        assertFalse(condition) & !condition\\
        assertNull(value) & value == null\\
        assertNotNull(value) & value != null\\
    \end{tabular}
\end{center}

\begin{center}
    \includegraphics[width=0.9\columnwidth]{pictures/testmethode.png}    
\end{center}

\subsubsection{Sichtbarkeit}
\begin{center}
    \begin{tabular}{ll}
        \rowcolor[RGB]{239,239,239} 
        \textbf{Keyword} & \textbf{Sichtbar für} \\ \hline
        public & Alle Klassen \\
        protected & Klassen im selben Package und abg. Klassen\\
        (keines) & Klassen im selben Package \\
        private & Nur eigene Klasse \\
    \end{tabular}
\end{center}

Für Zugriffe auf \verb|private|-Variablen können Getter- und Setter-Methoden definiert werden. Siehe auch \ref{GetSet}


\subsection{Generics}

\textit{Disclaimer, folgende Themen sind hier nicht enthalten: Typebounds, Wildcardtypen, Covarianz, Type-Erasure}

Generics dienen dazu, typ-unabhängige Frames zu erstellen. 

\subsubsection{Generische Variablen}
Die Variable dient als Platzhalter innerhalb einer generischen Klasse, Methode oder Interface. Sie wird mit \verb|<>| umschlossen 
und kann innerhalb der Klasse,... wie ein normaler Typ verwendet werden.\\
Häufig verwendete Namen:\\
\begin{tabular}{l l|l l}\hline
    E & Element & T & Type\\
    K & Key     & V & Value\\
    N & Number  & S, U ,V,... & 2ter, 3ter, 4ter Type \\\hline
\end{tabular}\\
\\
Mehrere Typ-Variablen gleichzeitig sind zulässig, z.B. \verb|<T, U>|

\subsubsection{Generische Klassen}
Eine Klasse mit Typ-Variable. Die Variable dient als Platzhalter für unbekannten Typ. Verschachtelung bei der Anwendung ist zulässig.\\
\includegraphics[width=\linewidth]{pictures/generic-class.jpg}\\
Bei der Anwendung macht der Compiler eine statische Prüfung des Datentyps. der Type-Cast entfällt (auto-boxing, -unboxing)


\subsubsection{Generische Interfaces}\label{StackIterator}
Gleiche Syntaxregeln wie bei generischen Klassen, jede Klasse kann generische Interfaces implementieren.\\
Anwendung:\\
\includegraphics[width=\linewidth]{pictures/generic-interface1.jpg}
\hrule
\includegraphics[width=\linewidth]{pictures/generic-interface2.jpg}

\subsubsection{Generische Methoden}
Generische Methoden sind unabhängig von generischen Klassen und Interfaces. Die Implementation kann in normalen, 
nicht generischen Klassen erfolgen und ist auch in statischen Methoden zulässig.\\
Die Typ-Variable muss in Klammer \verb|<>| vor dem Rückgabewert stehen. \\
\verb|public static <T> List<T> merge(List<T> a, List<T> b) { … }|

