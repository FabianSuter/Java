{\small
\section{Schnittstellen}
    \begin{tabular}{l}
        $\bullet$ Dient als Schleuse zwischen Klasse und Aussenwelt.\\
        $\bullet$ Die Klasse muss die Funktionalität \textit{implementieren}\\
        $\bullet$ Die Aussenwelt darf die Funktionalität \textit{nutzen}.\\\hline
        $\bullet$ Methode(n) einer Schnittstelle: implizit \verb|public| und \verb|abstract|.\\
        $\bullet$ Deshalb werden nur \textbf{Methodendeklarationen} aufgeführt,\\
        $\qquad$ alles andere ist ungültig/unnötig.\\\hline

        $\bullet$ Diese Taktik wird als \textbf{lose Kopplung} bezeichnet.\\
        $\bullet$ Erlaubt unabhängige Entwicklung verschiedener Teams.\\\hline

        $\bullet$ Mehrere Klassen können eine Schnittstelle implementieren\\
        $\bullet$ Klasse kann aber auch mehrere Schnittstellen implementieren\\
        $\qquad$ $\rightarrow$ \textbf{Mehrfach-Implementierung} erlaubt.\\
    \end{tabular}
    \vspace{-0.3cm}

\subsection{Abstrakte Klassen vs. Interfaces}
    \begin{tabularx}{\linewidth}{|X c X|} \hline
        \tikz[baseline=(text.base)]\node[fill=blue, fill opacity=0.3, text opacity=1, rounded corners, inner sep=2pt, minimum height=5pt] (text) {\textbf{Abstrakte Klassen}}; & & \tikz[baseline=(text.base)]\node[fill=green, fill opacity=0.3, text opacity=1, rounded corners, inner sep=2pt, minimum height=5pt] (text) {\textbf{Interfaces}}; \\
        (siehe auch \ref{AbstractClass} ) enthalten Instanzvariablen, Konstruktoren und teilweise implementierte Methoden & $\longleftrightarrow $ & enthalten nur Deklarationen, keinen Code \\
        \hline
    \end{tabularx}

    \begin{tabular}{l}
        \rowcolor[RGB]{239,239,239} 
        \textbf{Wann \tikz[baseline=(text.base)]\node[fill=green, fill opacity=0.3, text opacity=1, rounded corners, inner sep=2pt, minimum height=5pt] (text) {Interfaces};?}\\\hline
        $\bullet$ Implementierung (noch) nicht bekannt\\
        $\bullet$ Implementierungen haben wenig gemeinsamen Code\\
        $\bullet$ Losere Kopplung\\\hline
        \rowcolor[RGB]{239,239,239} 
        \textbf{Wann \tikz[baseline=(text.base)]\node[fill=blue, fill opacity=0.3, text opacity=1, rounded corners, inner sep=2pt, minimum height=5pt] (text) {abstrakte Klassen};?}\\\hline
        $\bullet$ Code bei mehreren Klassen wiederverwenden\\
        $\bullet$ Klassen haben gemeinsame Instanzvar. und Methoden\\
        $\bullet$ Konstruktor erforderlich, um Instanzvar. zu init.\\
    \end{tabular}
    \vspace{-0.3cm}

\subsection{Ein Interface - mehrere Implementierung}
    \vspace{-0.2cm}
    \begin{center}
        \begin{minipage}{0.4\columnwidth}
                % Code für Vehicle
                \lstinputlisting{code/interfaceVehicle.java}
        \end{minipage}
    \end{center}
    \vspace{-0.6cm}
    \begin{center}
        \begin{minipage}{0.47\columnwidth}% Code für RegularCar (links)
            \lstset{
                basicstyle=\scriptsize\ttfamily\color{purplecode}, % Kleinere Schriftgröße + Violett Stil
                keywordstyle=\bfseries\color{purplecode},
                identifierstyle=\color{purplecode},
            }
            \lstinputlisting{code/RegularCar.java}
        \end{minipage}
        \hspace{0.01\columnwidth} % Abstand für den Strich
        \vrule width 0.5pt % Vertikaler Strich
        \hspace{0.01\columnwidth} % Abstand für den Strich
        \begin{minipage}{0.47\columnwidth} % Code für RacingCar (rechts)
            \lstset{
                basicstyle=\scriptsize\ttfamily\color{turquoisecode}, % Kleinere Schriftgröße + Türkis Stil
                keywordstyle=\bfseries\color{turquoisecode},
                identifierstyle=\color{turquoisecode},
            }
            \lstinputlisting{code/RacingCar.java}
        \end{minipage}
    \end{center}
    \vspace{-0.2cm}

\subsection{Mehrere Interfaces - eine Implementierung}
    \vspace{-0.2cm}
    \begin{center}
        % Code für Vehicle (links)
        \begin{minipage}{0.49\columnwidth}
            \lstset{
                basicstyle=\scriptsize\ttfamily\color{purplecode}, % Kleinere Schriftgröße + Violett Stil
                keywordstyle=\bfseries\color{purplecode},
                identifierstyle=\color{purplecode},
            }
            \lstinputlisting{code/interfaceVehicle.java}
        \end{minipage}
        \hfill
        % Code für House (rechts)
        \begin{minipage}{0.49\columnwidth}
            \lstset{
                basicstyle=\scriptsize\ttfamily\color{turquoisecode}, % Kleinere Schriftgröße + Türkis Stil
                keywordstyle=\bfseries\color{turquoisecode},
                identifierstyle=\color{turquoisecode},
            }
            \lstinputlisting{code/interfaceHouse.java}
        \end{minipage}
    \end{center}
    \vspace{-0.9cm}
    \begin{center}
        \begin{minipage}{0.8\columnwidth}
            % MobileHome
            {\scriptsize
            \lstinputlisting{code/mobileHome.java}
            \vspace{-0.6cm}
            \textcolor{purplecode}{\lstinputlisting{code/mobileHome_1.java}}
            \vspace{-0.6cm}
            \textcolor{turquoisecode}{\lstinputlisting{code/mobileHome_2.java}}}
        \end{minipage}
    \end{center}
    \vspace{-0.6cm}
}