{\small
\section{Basics}

\subsection{Grundsätzlich: Java ...}

% Diese Zusammenfassung dient zur Hilfe beim Programmieren mit Java anhand von C++-Vorwissen. Es gibt einige wichtige Unterschiede, \textbf{Java:}
\begin{itemize}[itemsep=1pt, parsep=0pt]
    \item kennt keine Zeiger / Pointer
    \item kennt keine Funktionen (reine OO-Sprache)
    \item kennt kein Überladen von Operatoren
    \item kennt keine Destruktoren (Garbage-Collector gibt Speicher frei)
    \item kennt keine Mehrfachvererbung
    \item stellt eine umfangreiche Klassenbibliothek zur Verfügung
    \item -Programme sind plattformunabhängig
    \item ist weniger hardwarenah wie C++
\end{itemize}

\subsection{Java Runtime}
Der Compiler erzeugt anders als in C++ \textbf{Bytecode}, welcher anschliessend auf einer \textbf{J}ava \textbf{V}irtual \textbf{M}achine laufen kann.
$\Rightarrow$ Java \tikz[baseline=(text.base)]\node[fill=yellow, fill opacity=0.3, text opacity=1, rounded corners, inner sep=2pt, minimum height=5pt] (text) {plattformunabhängig};, 
da jede Plattform (MacOS) eine JVM hat.


\section{Datentypen}

Ein Datentyp besteht aus \textit{Werten} und \textit{Operationen}. In Java:\\
\vspace{-0.2cm}
\textbf{Primitive Datentypen} $\Leftrightarrow$ \textbf{Referenzdatentypen}.

\subsection{Primitive Datentypen}
    Plattformunabhängig gleicher Wertebereich, keine \verb|unsigned|-Typen.
    \vspace{-0.1cm} 

    \begin{center}
        \begin{tabular}{lll}
            \rowcolor[RGB]{239,239,239} 
            \textbf{Typ} & \textbf{Beschr.}        & \textbf{Beispiele} \\ \hline
            boolean      & Bool'scher Wert         & true, false \\
            char         & Textzeichen (UTF16)     & 'a', 'B', etc. \\
            byte         & Ganzzahl (8 Bit)        & -128 bis 127 \\
            short        & Ganzzahl (16 Bit)       & -32'768 bis 32'767 \\
            int          & Ganzzahl (32 Bit)       & $ -2^{31} $ bis $ 2^{31}-1 $ \\
            long         & Ganzzahl (64 Bit)       & $ -2^{63} $ bis $ 2^{63}-1 $, 1L \\
            float        & Gleitkommazahl (32 Bit) & 0.1f, 2e4f \\
            double       & Gleitkommazahl (64 Bit) & 0.1, 2e4 \\
        \end{tabular}
    \end{center}
    \vspace{-0.2cm}

    Gleitkommazahlen ohne Angaben sind automatisch \verb|double|.

    \vspace{-0.1cm} 

    \subsubsection{Überlauf / Unterlauf}
        \begin{tabular}{ll}
            \textbf{Ganzzahlen}         & 2147483647 + 1 $ \rightarrow $ -2147483648 \\ \hline
            \textbf{Gleitkommazahlen}   & 2 * 1e308 $ \rightarrow $ \verb|POSITIVE_INFINITY|\\
            & 5e-324 / 2 $ \rightarrow $ 0.0\\
        \end{tabular}
        \vspace{-0.3cm}

    \subsubsection{Text-Literale}
        \begin{tabular}{l}
            \textbf{char} mit Apostrophen: \verb|'A'|, \verb|'\n'| (NewLine), \verb|'\''| (Apostroph) \\
            \textbf{String} mit Anführungszeichen: \verb|"Say \"hello\"!\n"|
        \end{tabular}
        \vspace{-0.3cm} 

    \subsubsection{Undefinierte Operationen}
        \begin{tabular}{l}
            \textbf{Ganzzahlen} werfen bei Division/Modulo durch 0 einen\\
            \tikz[baseline=(text.base)]\node[fill=red, fill opacity=0.2, text opacity=1, rounded corners, inner sep=2pt, minimum height=5pt] (text) {Fehler}; bzw. eine \verb|null|-Referenz generiert eine \tikz[baseline=(text.base)]\node[fill=red, fill opacity=0.2, text opacity=1, rounded corners, inner sep=2pt, minimum height=5pt] (text) {Exception};\\ \hline
            \textbf{Gleitkommazahlen} werfen bei Division durch 0 ein\\
            \tikz[baseline=(text.base)]\node[fill=red, fill opacity=0.2, text opacity=1, rounded corners, inner sep=2pt, minimum height=5pt] (text) {\verb|POSITIVE_INFINITY|}; resp. \tikz[baseline=(text.base)]\node[fill=red, fill opacity=0.2, text opacity=1, rounded corners, inner sep=2pt, minimum height=5pt] (text) {\verb|NEGATIVE_INFINITY|};\\ \hline
            Bei undefinierten Rechnungen wie 0 / 0 wird \tikz[baseline=(text.base)]\node[fill=red, fill opacity=0.2, text opacity=1, rounded corners, inner sep=2pt, minimum height=5pt] (text) {\verb|NaN|}; zurückgegeben.\\
        \end{tabular}
        \vspace{-0.3cm}
    
\subsection{Referenzdatentypen}

    \subsubsection{null-Referenz}
        Spezielle Referenz auf "kein Objekt", vordefinierte Konstante, welche
        \tikz[baseline=(text.base)]\node[fill=yellow, fill opacity=0.3, text opacity=1, rounded corners, inner sep=2pt, minimum height=5pt] (text) {für alle Referenztypen gültig}; ist. \textit{Dereferenzieren} der
        \verb|null|-Referenz generiert eine \tikz[baseline=(text.base)]\node[fill=red, fill opacity=0.3, text opacity=1, rounded corners, inner sep=2pt, minimum height=5pt] (text) {\verb|NullPointerException|};
        \vspace{-0.3cm}

    \subsubsection{Klassen: \textit{Siehe Kapitel \ref{Klassen}}}

    \subsubsection{Arrays}
        Arrays speichern wie in C++ mehrere Elemente mit selbem Datentyp. Der Zugriff erfolgt über Index, die Elemente liegen nebeneinander im Speicher.
        Die Grösse des Arrays muss bei der Deklaration festgelegt werden und ist später nicht mehr änderbar.\\
        Die Anzahl der Elemente kann in Java direkt mit \verb|length| abgerufen werden, siehe auch \textbf{\ref{Array-Loop}}.\\
        Falls der Index ungültig ist, wird eine\\
        \tikz[baseline=(text.base)]\node[fill=red, fill opacity=0.3, text opacity=1, rounded corners, inner sep=2pt, minimum height=5pt] (text) {\textbf{ArrayIndexOutOfBoundsException}}; geworfen.
        \vspace{-0.3cm}

    \subsubsection{Enums}
        Ähnliche Verwendung wie in C++. Enums können als Konstantenlisten verwendet werden:
        \lstinputlisting{code/Weekday.java}
        \vspace{-0.3cm}

    \subsubsection{Collections: \textit{Siehe auch Kapitel \ref{Collections}}}
        \begin{center}
            \begin{tabular}{lll}
                \rowcolor[RGB]{239,239,239} 
                \textbf{Interface} & \textbf{Klassen} & \textbf{Eigenschaften} \\ \hline
                List & ArrayList, LinkedList & Folge von Elementen, \\
                    &                       & Duplikate möglich\\
                \hline
                Set  & HashSet, TreeSet & Menge von Elementen, \\
                    &                       & keine Duplikate \\
                \hline
                Map  & HashMap, TreeMap & Abbildung Schlüssel \\
                    &                  & $\hookrightarrow$ Werte, \\
                    &                  & keine Duplikate \\
            \end{tabular}
        \end{center}
        \vspace{-0.6cm}

\subsection{Typumwandlung}
    \begin{minipage}{0.4\columnwidth}
        \textbf{Implizit:}\\
        Klein zu Gross,\\
        kein Cast-Operator erf.\\
        \lstinputlisting{code/implizit.java}
    \end{minipage}
    \begin{minipage}{0.55\columnwidth}
        \textbf{Explizit:}\\
        Gross zu Klein,\\
        Cast-Operator erf.\\
        \lstinputlisting{code/explizit.java}
    \end{minipage}
    \vspace{-0.3cm} 
}
