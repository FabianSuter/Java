\section{Basics}

\subsection{Grundsätzlich}

Diese Zusammenfassung dient zur Hilfe beim Programmieren mit Java anhand von C++-Vorwissen. Es gibt einige wichtige Unterschiede, \textbf{Java:}

\begin{itemize}[itemsep=1pt, parsep=0pt]
    \item kennt keine Zeiger / Pointer
    \item kennt keine Funktionen (reine OO-Sprache)
    \item kennt kein Überladen von Operatoren
    \item kennt keine Destruktoren (Garbage-Collector gibt Speicher frei)
    \item kennt keine Mehrfachvererbung
    \item stellt eine umfangreiche Klassenbibliothek zur Verfügung
    \item -Programme sind plattformunabhängig
    \item ist weniger hardwarenah wie C++
    \item ...
\end{itemize}

\subsection{Sourcecode Hello world Java}

\lstinputlisting{code/HelloWorld.java}

% TODO: In Schleifen aufführen
% \subsection{For each Loop}\label{forEach}

% \verb|For each| - Loops laufen automatisch ein Array ab, ohne das zuerst die Grösse ermittelt werden muss. 
% Die C - \verb|for| - Loop Syntax wurde etwas erweitert.
% Syntax:

% \lstinputlisting{code/forEach.cpp}

% Siehe: Als Arrayname wird ein \say{Pointer} auf das Array verlangt, nicht auf das erste Element!  
% Ein \verb|For each| funktioniert \textbf{nicht} mit einem \textbf{Pointerpointer}. 


\subsection{Java Runtime}
Der Compiler erzeugt anders als in C++ \textbf{Bytecode}, welcher anschliessend auf einer Java Virtual MAchine (JVM) laufen kann.
Dadurch wird Java plattformunabhängig, da jede Plattform eine JVM hat, sei es Windows, Android oder MacOS.


\section{Datentypen}

Ein Datentyp besteht aus \textit{Werten} und \textit{Operationen}. Java besitzt zwei generelle Datentypen, 
namentlich \textbf{Primitive Datentypen} und \textbf{Referenzdatentypen}.

\subsection{Primitive Datentypen}
Sie haben plattformunabhängig den gleichen Wertebereich, es gibt keine \verb|unsigned|-Typen. Für bool'sche Werte gibt es den Datentyp \verb|boolean|.

\begin{center}
    \begin{tabular}{lll}
        \rowcolor[RGB]{239,239,239} 
        \textbf{Typ} & \textbf{Beschr.}        & \textbf{Beispiele} \\ \hline
        boolean      & Bool'scher Wert         & true, false \\
        char         & Textzeichen (UTF16)     & 'a', 'B', etc. \\
        byte         & Ganzzahl (8 Bit)        & -128 bis 127 \\
        short        & Ganzzahl (16 Bit)       & -32'768 bis 32'767 \\
        int          & Ganzzahl (32 Bit)       & $ -2^{31} $ bis $ 2^{31}-1 $ \\
        long         & Ganzzahl (64 Bit)       & $ -2^{63} $ bis $ 2^{63}-1 $, 1L \\
        float        & Gleitkommazahl (32 Bit) & 0.1f, 2e4f \\
        double       & Gleitkommazahl (64 Bit) & 0.1, 2e4 \\
    \end{tabular}
\end{center}

Gleitkommazahlen ohne Angaben sind automatisch \verb|double|.

\subsubsection{Überlauf / Unterlauf}
Bei \textbf{Ganzzahlen} ist der Überlauf in Java definiert, im Gegensatz zu C++.\\
2147483647 + 1 $ \rightarrow $ -2147483648\\

Bei \textbf{Gleitkommazahlen} gilt dasselbe:\\
2 * 1e308 $ \rightarrow $ \verb|POSITIVE_INFINITY|\\
5e-324 / 2 $ \rightarrow $ 0.0

\subsubsection{Undefinierte Operationen}
\textbf{Ganzzahlen} werfen bei Division/Modulo durch 0 einen Fehler bzw. eine Exception.\\

\textbf{Gleitkommazahlen} werfen bei Division durch 0 ein \verb|POSITIVE_INFINITY| resp. \verb|NEGATIVE_INFINITY|\\
Bei undefinierten Rechnungen wie 0 / 0 wird \verb|NaN| zurückgegeben.

\subsubsection{Text-Literale}
\textbf{char} mit Apostrophen: \verb|'A'|, \verb|'\n'| (NewLine), \verb|'\''| (Apostroph) \\

\textbf{String} mit Anführungszeichen: \verb|"Say \"hello\"!\n"|

\subsection{Referenzdatentypen}

\subsubsection{Arrays}
Arrays speichern wie in C++ mehrere Elemente mit selbem Datentyp. Der Zugriff erfolgt über Index, die Elemente liegen nebeneinander im Speicher.
Die Grösse des Arrays muss bei der Deklaration festgelegt werden und ist später nicht mehr änderbar.\\
Die Anzahl der Elemente kann in Java direkt mit \verb|length| abgerufen werden, siehe auch \textbf{\ref{Array-Loop}}.\\
Falls der Index ungültig ist, wird eine\\
\textbf{ArrayIndexOutOfBoundsException} geworfen.

\subsection{Typumwandlung}

\begin{minipage}{0.4\columnwidth}
    \textbf{Implizit:}\\
    Klein zu Gross,\\
    kein Cast-Operator erf.\\
    \lstinputlisting{code/implizit.java}
\end{minipage}
\begin{minipage}{0.55\columnwidth}
    \textbf{Explizit:}\\
    Gross zu Klein,\\
    Cast-Operator erf.\\
    \lstinputlisting{code/explizit.java}
\end{minipage}

